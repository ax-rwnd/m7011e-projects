\task{CodeExplorer}
\begin{refsection}
Exploring and discovering things can be really motivating, especially if that you're discovering is interesting and well made. With this in mind, one could think of using the concept to motivate people to learn- or practice things that might not otherwise be very interesting to them. One such subject is programming; Given how the programming skill is turning into a necessity among techical workers, it's important that everyone learns it. Unfortunately, a great deal of people find it really hard to get started with it, which is where assignments like this start making sense.

To create assignments like this, it's prudent to have a supporting framework that takes care of most technical details. In fact, comparatively few lecturers actually know how to program, so it would be unreasonable to ask lecturers to create something like this for themselves. Thus, an easily extensible framework with some of the most common bells and whistles would be a good start. Consider an overmap with various objectives related to the subject with power-ups and trials along the way. Couple this with some sort of narrative to provide context for said objectives and suddenly you have an informative game.

Your task is to design an engine for exploiting this in whatever field one could seem fit. The maintainers should be able to setup systems that fit their use-cases with some prior knowledge and a considerable amount of time, say 40 working hours. Obviously, some demo scenario should also be written up to properly show off what the system is able to do. However, this scenario needen't be very creative, as long as it properly showcases the entirety of the system.

\subsection*{Requirements}
For an engine like this to be successful, you will (apart from the regular requirements) at least need to implement:
\begin{itemize}
    \item An overmap construct or some other way of describing connected problems.
    \item Some system of game mechanics that implements subject-related tasks.
    \item Development tools that provide a intuitive interface for content creators.
    \item A sample `campaign' that properly demonstrates the features of your system.
\end{itemize}
If that's not enough for you, consider implementing
\begin{itemize}
    \item A social feature that includes code-review.
    \item Group tasks and games wherein students need to cooperate to complete tasks.
    \item Anonymous leaderboards (consider motivators and demotivators).
\end{itemize}

\end{refsection}
