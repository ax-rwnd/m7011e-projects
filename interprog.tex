\task{InterProg}
\begin{refsection}
    In this task you're supposed to write a game for hearts (socializers) wherein players meet and interact with other players through code. The core of this should be centered around making people cooperate and push one another to finishing challenges. When figuring out the game mechanics, you may want to read up on some examples\supercite{wiki:socialgames} of popular social network games from which you may want to draw inspiration.

%The game mechanics are largely free for you to decide, but should incorporate social- and cooperative elements. One could imagine a game wherein the players are part of a company and are given software development tasks to perform. The results themselves might not even be formally evaluated, but instead showcased in a player portfolio.

   When implementing the game, remember that you should design the challenges to teach the players how to write code. In many cases, you find yourself looking for help or solutions to specific issues and sometimes you find that your solution is good enough to help ohers. Thus, you may want to investigate the viability of various programming elements to act as motivators for your tasks, including the use the products of their work as both as solutions and as usable/collectable artifacts.
    
%    As with many games, it's not impossible that the end result could be realized as an actual product and may have some people maintaining it full-time. Furthermore, the scalability of each game is bounded by the size of the group participating, so the game itself might only need to be able to scale with more game servers. Keep this in mind when developing your solution.

\subsection*{Requirements}
Apart from the general requirements you'll need to implement:
\begin{itemize}
    \item Game elements that feature social coding.
    \item A game structure where users/players socialize and cooperate to help themselves.
\end{itemize}
Should you want to expand on the concept you may also want to:
\begin{itemize}
    \item Create some sort of group-based challenge to intertwine the users.
    \item Make the game a real-time experience using WebSockets or similar components.
    \item Open the service up for integration to social networks.
\end{itemize}

\printbibliography[heading=subbibliography]
\end{refsection}
