\task{Luttis}
\begin{refsection}
Programming contests are one way to help students learn problem solving and programming. The idea is that assignments are handed out to the participants who solve the problem by turning in a program that executes the solution. An automated system then performs a set of tests on the program to test the validity before judging the solution as correct or incorrect.

The major issue with this kind of system is that it primarily aims to inspire students that are already motivated to go further. This is mostly done by supplying scoreboards and comparisons for everyone to see, which isn't very encouraging if you did poorly.

Now, your task is to implement a contest system like this, but instead of rewarding the winners, you should reward the learners. If done correctly, you will hopefully no longer be feeding the insatiable hunger for beating your opponents, but instead the (if somewhat hidden) insatiable hunger of self-actualization. Some ideas that have been identified for this purpose are:
\begin{itemize}
    \item Provide small steps and let users work at their level.
    \item Consider employing some creative peer-review exercise.
    \item Only use positive progress, do not punish failures with deductions.
    \item Consider having a roster of problems to solve from which participants may cherry-pick, rather than using a fixed set.
    \item Provide small rewards and let the users redeem the rewards (new avatars, color schemes, online editor upgrades).
    \item If you have a scoreboard, keep it anonymous and possibly limited to showing slices around your rank. 
\end{itemize}

    When implementing this system, try to keep scalability in mind. At first there may only be a single competition or course using this system at a time, but if the system becomes popular, it should absolutely be possible to increase the capacity to handle the load. Furthermore, the system can be assumed to be run by volunteers, i.e.\ contributors without pay, so while only the competition/course maintainers need to spend time keeping the system up, the code itself should be well documented and easy to navigate.

\subsection{Requirements}
To complete this assignment, you'll have to implement, on top of the regular requirements, a system that
\begin{itemize}
    \item Corrects programs, either by testing input/output or by unit testing (or some other, strict way).
    \item Encourages weaker students to start learning.
    \item Employs at least one of the strategies mentioned above.
\end{itemize}
Should you want to do more, consider
    \begin{itemize}
        \item Implementing administrative tools for better management.
        \item Building a platform for managing a service-global corpus of assignments that can be used in these courses and competitions.
    \end{itemize}
\end{refsection}
