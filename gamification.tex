\section*{Gamification}
%\thispagestyle{empty}
\begin{refsection}
\fancyhead[R]{Theory: \currentname}
There are many approaches to- and uses of gamification that are being tested and deployed today ranging from reward-based platforms to company-wide running campaigns to illicit customer manipulation. The reasons for employing gamification is usually either that users need some external motivation to get started, or that uses have been identified where better results can be reached by constantly feeding this kind of motivation.

    Some concrete examples of gamification at work is the Steam gaming platform, where users are constantly encouraged to show their progress by achievements and their pages; Youtube, which autoplays videos for their users encourage them to stay and services like StackOverflow and Reddit where users get counters and medals for being helpful or participating. In large, these elements may seem obvious or ubiquitous, but in reality these tools are very much aimed at making users participate more. So, instead of killing trees explaining things that are best researched individually, have a look Gamified.UK's list of gamification\supercite{gamifieduk} elements to get started and look into the elements that you think are interesting.

    One theory that is good to know about is Bartle's Taxonomy\supercite{bartle}. It states that there are four types of players: Hearts (socializers), clubs (slayers), diamonds (achievers) and spades (explorers). Hearts are the ones that mainly play games to socialize and hang out with people, these people can often be engaged by providing cooperative and social elements where players interact. Then there are the clubs who like to compete with others, they are usually engaged by elements that let them beat their opponents or show that they are the best. After that there are the diamonds who like to simply complete everything; Be it maxed completion rates, large lists badges or pieces of map discovered, they want something to quantify their completion. Last, but not least, there are the spades who like to explore things; Be it intricate game mechanics, vast game worlds or well-guarded secrets, this is their `thing'. These all have some defining characteristics that may be designed towards, but it's hard to design a system that encompasses all, so think of this while designing your service.

\printbibliography[heading=subbibliography]

\end{refsection}

