\task{Re-course}
A popular tool amongst programmers is StackOverflow (Discourse), which is a forum where you can ask- and answer questions. This is great for programming in general as you can find solutions to most common problems that you may run into. However, it's less great from a learning perspective, since it encourages people to turn to the internet before trying to solve the problem themselves.

If you were to introduce a course-local StackOverflow alternative, people would be able to create a database of common lab-related problems that might not otherwise be covered. With such a system in place, people will be encouraged to explain problems to their peers, helping not only their colleagues learn, but also themselves. They would also be able to discuss and clarify issues that arise during the course, with lab supervisors and lecturers moderating the forum and making sure that accepted answers actually are correct.

One could imagine that a system like this would be run by the university IT Support team and be used by a great deal of courses, lecturers and students across the university. This puts some weight on the scalability of the system, since there may be several thousands of users discussing several hundreds of courses. Thus, more concretely, your assignment is to create a website that works like StackOverflow and is able to be scaled to fit a userbase of approximately 15000 users.
