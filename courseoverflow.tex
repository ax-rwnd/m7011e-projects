\task{CourseOverflow}
\begin{refsection}
A popular tool amongst programmers is StackOverflow, which is a forum where you can ask- and answer questions. This is great for programming in general as you can find solutions to most common problems that you may run into. However, it's less great from a learning perspective, since it encourages people to turn to the internet before trying to solve the problem themselves.

If you were to introduce a course-local StackOverflow alternative, people would be able to create a database of common course-related problems that might not otherwise be covered. With such a system in place, people will be encouraged to explain problems to their peers, helping not only their colleagues learn, but also themselves. They would also be able to discuss and clarify issues that arise during the course, with lab supervisors and lecturers moderating the forum and making sure that accepted answers actually are correct.

Another advantage would be that course-local rewards for answering questions could be implemented, meaning that people would get external motivation to take the time to really understand the subject by explaining it to someone else. In fact, one of the major tools to get people hooked on services like this is to actually provide some quantifiable metric of their participation and recognition within the community. This can be done in many ways with some of the most common tools being posts made; Thanks- or likes received and having posts pinned- or recognized by moderators. Taking that further would be to recognize people as valued members of the community by marking users whose posts are remarkably good by some special tag or banner, or even promoting members to moderators once they've proven themselves competent.

One could imagine that a system like this would be run by the university IT Support team and be used by a great deal of courses, lecturers and students across the university. This puts some weight on the scalability of the system, since there may be several thousands of users discussing several hundreds of courses. Thus, more concretely, your assignment is to create a website that works like StackOverflow, is gamified and is able to be scaled to fit a larger userbase. In addition, the site needs proper tools for administration and course management to keep track of courses, teachers and students.

\subsection*{Requirements}
More concretely, apart from the general requirements you will need to implement
\begin{itemize}
    \item Some server architecture that is able to coordinate multiple courses.
    \item A general forum system that is independent on the target course.
    \item Gamification elements that motivate people to contribute more.
    \item Moderation tools for the community of the system.
    \item Administration for managing courses.
\end{itemize}
Should you find that you want more work you may want to implement
\begin{itemize}
    \item Some kind of reward system for answering questions.
    \item A common goal indicator for question in the course.
    \item Forum-champion statuses that in theory could be used to promote lab assistants.
    \item Legacy holdover mechanisms that keep (relevant) fragments of the previous iteration in the system.
    \item If possible, canvas integration.
\end{itemize}

\end{refsection}
