\task{Bug Hunter}
\begin{refsection}
    Fourth and last of the player types in Bartle's Taxonomy is the diamond (achiever). As the name implies, achieving high scores and finishing every last bit of games is a major driving force for these players. In fact, it's not uncommon to see achievers bragging about their 100\% completion stats in esoteric games.

    Oddly enough, a fairly common task within software engineering already hits this spot by design, and that is unit testing. As you should know by now, unit testing is the first and probably most important building block when testing code. If your code is tested properly, you can guarantee that it works fine when you assemble it with other code, as it promises to satisfy your conditions.

    Furthermore, unit tests are easily quantifiable by their code coverage. That is, the ratio of code tested versus code untested. One could imagine a game wherein legacy code in various states of disarray would have to be unit tested. Depending on the amount of code coverage the player would be able to reach, the player would get a higher spot on a scoreboard.

    An interesting aspect of this assignment is that there are many old codebases that need testing and that nobody would considering going near. One could even imagine that companies could donate legacy code that needs testing that they don't actually want to test and let eager players try to reach 100\% test coverage.

    \subsection*{Requirements}
    For this assignment, apart from the regular requirements, you should implement
    \begin{itemize}
        \item A site that integrates multiple unit-testing tools.
        \item Automatic generation of coverage reports.
        \item Some mechanic for measuring your results with others.
    \end{itemize}
    If you find that you want more work, consider adding
    \begin{itemize}
        \item Ways for donors to submit code for testing.
        \item Achievements for special feats.
        \item Visualization of tested paths.
    \end{itemize}

\printbibliography[heading=subbibliography]
\end{refsection}
